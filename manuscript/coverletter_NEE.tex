\documentclass[ucm,12pt]{ucletter}

\setlength\parindent{1cm}

\usepackage{setspace}
\usepackage{graphicx}

%\name{Justin D. Yeakel}
\telephone{(209)~285-9571}
\email{jyeakel@ucmerced.edu}

\begin{document}
\begin{letter}{
  \\Nature Publishing Group\\
  The Macmillan Building\\
  4 Crinan Street\\ 
  London N1 9XW\\
  United Kingdom\\ \\
    \centerline{\bf{Re: Diverse interactions and ecosystem engineering stabilize community assembly}}
    \vspace{10mm}
}


\opening{To the Editorial Board at \emph{Nature Ecology \& Evolution},}

\setstretch{1.25}

%Intro
Please find attached the manuscript entitled "Diverse interactions and ecosystem engineering stabilize community assembly" co-authored by Justin Yeakel, Mathias Pires, Marcus de Aguiar, James O'Donnell, Paulo Guimaraes Jr., Dominique Gravel, and Thilo Gross, which we would like to submit for publication in \emph{Nature Ecology \& Evolution}. 

A longstanding question in ecology concerns the role of species interactions in determining the stability of ecological systems. Importantly, species interact not only directly with each other but indirectly through environmental effects. However the role of these ecosystem engineers has not yet been considered in models of ecological networks. Here we model the assembly of an ecological network where nodes represent ecological entities, including engineering species, non-engineering species, and the effects of the former on the environment, which we call abiotic modifiers. The links of the network that connect both species and modifiers represent trophic, service, and engineering dependencies, allowing us to evaluate the role of such multitype interactions - directly between species and indirectly through environment - on the assembly of ecological systems.

The results of our investigation show that increasing the proportion of ecosystem engineers within a community has nonlinear effects on observed extinction rates. While we find that a low amount of engineering increases extinction rates, a high amount of engineering has the opposite effect, and that the inclusion of engineering modifies the effects of service interactions on community robustness. Service interactions in many cases are the foundation of mutualisms, and we show that while high frequencies of mutualisms erode robustness in non-engineered systems, these negative effects are significantly reduced in highly engineered systems. Finally we show that redundancies in engineered effects promote community diversity by lowering the barriers to colonization. 

Overall we suggest that interactions where species affect others by altering the environment in a lasting way have been neglected in our understanding of community dynamics - especially through the lens of ecological networks. Yet such interactions may have profound effects on assembling communities over both ecological and evolutionary timescales.

Our work integrates a number of perspectives and sub-disciplines in ecology that should appeal to a broad readership. We expect this submission to be of particular interest to those using network approaches to explore ecological systems, in part due to our integration of both biotic and abiotic elements and the interactions between them. Moreover our work emphasizes the roles of trophic and mutualistic interactions, the dynamics of community assembly, and the role of ecosystem engineers in complex ecosystems.


\vspace{0mm}

\singlespacing
\closing{Sincerely,\\
\fromsig{\includegraphics[scale=0.2]{signature.jpg}}\\
\fromname{
Justin D. Yeakel\\
Assistant Professor\\
University of California, Merced}
}

We suggest the following reviewers:
\begin{itemize}
\item Barbara Drossel, Darmstadt University of Technology, drossel@fkp.tu-darmstadt.de
\item Douglas Erwin, Smithsonian NMNH, erwind@si.edu
\item Michel Loreau, Centre National de la Recherche Scientifique, michel.loreau@sete.cnrs.fr
\item Daniel Stouffer, School of Biological Sciences, University of Canterbury, daniel.stouffer@canterbury.ac.nz
\end{itemize}


\end{letter}
\end{document}
