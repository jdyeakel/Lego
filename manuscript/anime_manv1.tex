\documentclass[twocolumn,preprintnumbers,amsmath,amssymb,superscriptaddress]{revtex4}
%\usepackage[pdftex]{graphicx}

\usepackage{amsmath,amsfonts,amssymb}
\usepackage[english]{babel}
\usepackage[latin1]{inputenc}
\usepackage[T1]{fontenc}
\usepackage{color}
\usepackage{float}
\usepackage{verbatim}
\usepackage{graphicx}
\usepackage{bm}
\usepackage{mathtools}
\usepackage{stmaryrd}
\usepackage{anyfontsize}


%\usepackage{epstopdf}
%\usepackage{array}
%\usepackage{tabularx}
%\usepackage{multirow}
\usepackage{color}
%\usepackage{multibox}
%\usepackage{rotating}
%\usepackage{lineno}
%\usepackage[left]{lineno}
%\usepackage[comma,sort&compress]{natbib}
%\usepackage{authblk}
%\usepackage{multicol}

%\bibliographystyle{ieeetr}


%\linenumbers
%\setlength\linenumbersep{3pt}

\begin{document}



\author{Justin D. Yeakel} \affiliation{School of Natural Sciences, University
  of California, Merced, Merced, CA 95340, USA}

\author{Mathias Pires} \affiliation{}

\author{James O'Donnell} \affiliation{}

\author{Marcus de Aguiar} \affiliation{}

\author{Paulo Guimar\~aes Jr} \affiliation{}

\author{Dominique Gravel} \affiliation{}

\author{Thilo Gross} \affiliation{}

%\title{Simple rules yield complex communities: deconstructed species interactions and the assembly of communities}
%\title{Community assembly and dynamics by the deconstruction of species interactions}
\title{Quantization of ecological interactions yields insights into community assembly and dynamics}
%\author{Justin D. Yeakel${}^{1,2,*}$, Christopher P. Kempes${}^{2}$, \& Sidney Redner${}^{2,3}$ \\ \\
%${}^1$School of Natural Science, University of California Merced, Merced, CA \\
%${}^2$The Santa Fe Institute, Santa Fe, NM \\
%${}^3$Department of Physics, Boston University, Boston MA \\
%${}^*$To whom correspondence should be addressed: jdyeakel@gmail.com
%}


\begin{abstract}
abstract goes here
\end{abstract}

\maketitle

\section*{Introduction}

Amazing words. The best words.


\section*{Results}

{\bf Effects of engineers on community richness} 
% Negative effects of engineering at the scale of the community
% The effect of engineers on extinction cascade size
Increasing the number of engineers (species with $m \leftrightarrow n$ interactions with their respective objects) at time $t$ results in the potential for larger extinction cascades at time $t+1$, and this correlation increases with the number of engineers in the community.
This positive correlation between the number of objects at time $t$ and extinction cascade size at time $t+1$ results from the increasing interconnectedness that results from the higher number of objects relative to species in the system.
Because the existence of a given object is tied to the species that makes them (one or multiple), the effects of primary extinctions are magnified.
This correlation saturates as the proportion of engineers -- object to species ratio -- increases, and this MAY be due to redundant engineering.

% The effect of engineering colonization facilitation
Object richness is strongly positively correlated with the number of potential colonizers.
So from a species packing perspective, engineering enables successful colonization by increasing niche space.

{\bf Community invasibility} 
Niche space psitively correlated with number of possible invaders.
Fluctuates over time.
By default it is pulled down as invaders establish.
The number of potential invaders (i.e. niche space) is concave parabolic.
Before colonization takes hold, only primary producers can invade.
As additional species populate the community, invasibility grows to a maximum and then declines as the community fills and the number of potential colonizers diminishes.


\end{document}
