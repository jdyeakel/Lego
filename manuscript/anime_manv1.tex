\documentclass[twocolumn,preprintnumbers,amsmath,amssymb,superscriptaddress]{revtex4}
%\usepackage[pdftex]{graphicx}

\usepackage{amsmath,amsfonts,amssymb}
\usepackage[english]{babel}
\usepackage[latin1]{inputenc}
\usepackage[T1]{fontenc}
\usepackage{color}
\usepackage{float}
\usepackage{verbatim}
\usepackage{graphicx}
\usepackage{bm}
\usepackage{mathtools}
\usepackage{stmaryrd}
\usepackage{anyfontsize}


%\usepackage{epstopdf}
%\usepackage{array}
%\usepackage{tabularx}
%\usepackage{multirow}
\usepackage{color}
%\usepackage{multibox}
%\usepackage{rotating}
%\usepackage{lineno}
%\usepackage[left]{lineno}
%\usepackage[comma,sort&compress]{natbib}
%\usepackage{authblk}
%\usepackage{multicol}

%\bibliographystyle{ieeetr}


%\linenumbers
%\setlength\linenumbersep{3pt}

\begin{document}



\author{Justin D. Yeakel} \affiliation{School of Natural Sciences, University
  of California, Merced, Merced, CA 95340, USA}

\author{Mathias Pires} \affiliation{}

\author{James O'Donnell} \affiliation{}

\author{Marcus de Aguiar} \affiliation{}

\author{Paulo Guimar\~aes Jr} \affiliation{}

\author{Dominique Gravel} \affiliation{}

\author{Thilo Gross} \affiliation{}

%\title{Simple rules yield complex communities: deconstructed species interactions and the assembly of communities}
%\title{Community assembly and dynamics by the deconstruction of species interactions}
\title{Quantization of ecological interactions yields insights into community assembly and dynamics}
%\author{Justin D. Yeakel${}^{1,2,*}$, Christopher P. Kempes${}^{2}$, \& Sidney Redner${}^{2,3}$ \\ \\
%${}^1$School of Natural Science, University of California Merced, Merced, CA \\
%${}^2$The Santa Fe Institute, Santa Fe, NM \\
%${}^3$Department of Physics, Boston University, Boston MA \\
%${}^*$To whom correspondence should be addressed: jdyeakel@gmail.com
%}


\begin{abstract}
abstract goes here
\end{abstract}

\maketitle

\section*{Introduction}

Amazing words. The best words.


\section*{Results}

{\bf Effects of engineers on community richness} 
% Negative effects of engineering at the scale of the community
% The effect of engineers on extinction cascade size
Increasing the number of engineers (species with $m \leftrightarrow n$ interactions with their respective objects) at time $t$ results in the potential for larger extinction cascades at time $t+1$, and this correlation increases with the number of engineers in the community.
This positive correlation between the number of objects at time $t$ and extinction cascade size at time $t+1$ results from the increasing interconnectedness that results from the higher number of objects relative to species in the system.
Because the existence of a given object is tied to the species that makes them (one or multiple), the effects of primary extinctions are magnified.
This correlation saturates as the proportion of engineers -- object to species ratio -- increases, and this MAY be due to redundant engineering.

% The effect of engineering colonization facilitation
Object richness is strongly positively correlated with the number of potential colonizers.
So from a species packing perspective, engineering enables successful colonization by increasing niche space.

{\bf Community invasibility} 
In the context of community assembly, the niche space that is created by a given assemblage can be measured by evaluating the number of potential colonizers that can \emph{fit} within the assemblage from the species pool.
Accordingly, with each successful colonization, the number of potential colonizers declines with the smaller species pool.
Increases in the number of potential colonizers after a successful colonization event is thus due to niche space created by the successful colonizer.

We find that the proportion of the species pool that are capable of successfully colonizing the assembling community is concave parabolic, and this makes intuitive sense.
First, the proportion of potential colonizers is equal to the proportion of species that are primary producers that do not have too many `need' dependencies.
As the community assembles, niche space expands to a maximum, and then declines as the species pool diminishes and the community is filled.

%Over need_threshold
The ANIMe model 

%Over make_threshold
For communities with a large proportion of ecosystem engineers, niche expansion and constraction is similarly parabolic, but much more variable.
The proportion of initial colonizers is generally lower because there are many more indirect dependencies on engineers


\end{document}
