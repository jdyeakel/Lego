\documentclass[onecolumn,preprintnumbers,amsmath,amssymb,superscriptaddress]{revtex4}
%\usepackage[pdftex]{graphicx}

\usepackage{amsmath,amsfonts,amssymb}
\usepackage[english]{babel}
\usepackage[latin1]{inputenc}
\usepackage[T1]{fontenc}
\usepackage{color}
\usepackage{float}
\usepackage{verbatim}
\usepackage{graphicx}
\usepackage{bm}
\usepackage{mathtools}
\usepackage{stmaryrd}
\usepackage{anyfontsize}


\graphicspath{{../Anime/figures/}}


%\usepackage{epstopdf}
%\usepackage{array}
%\usepackage{tabularx}
%\usepackage{multirow}
\usepackage{color}
%\usepackage{multibox}
%\usepackage{rotating}
%\usepackage{lineno}
%\usepackage[left]{lineno}
%\usepackage[comma,sort&compress]{natbib}
%\usepackage{authblk}
%\usepackage{multicol}

%\bibliographystyle{ieeetr}


%\linenumbers
%\setlength\linenumbersep{3pt}

\begin{document}



\author{Justin D. Yeakel} \affiliation{School of Natural Sciences, University
  of California, Merced, Merced, CA 95340, USA}

\author{Mathias Pires} \affiliation{}

\author{James O'Donnell} \affiliation{}

\author{Marcus de Aguiar} \affiliation{}

\author{Paulo Guimar\~aes Jr} \affiliation{}

\author{Dominique Gravel} \affiliation{}

\author{Thilo Gross} \affiliation{}

%\title{Simple rules yield complex communities: deconstructed species interactions and the assembly of communities}
%\title{Community assembly and dynamics by the deconstruction of species interactions}
\title{Quantization of ecological interactions yields insights into community assembly and dynamics}
%\author{Justin D. Yeakel${}^{1,2,*}$, Christopher P. Kempes${}^{2}$, \& Sidney Redner${}^{2,3}$ \\ \\
%${}^1$School of Natural Science, University of California Merced, Merced, CA \\
%${}^2$The Santa Fe Institute, Santa Fe, NM \\
%${}^3$Department of Physics, Boston University, Boston MA \\
%${}^*$To whom correspondence should be addressed: jdyeakel@gmail.com
%}

\section{Brief description of extinction due to similarity in resource overlap}

Extinction probabilities are based on a similarity metric, where overlap in food resources and need resources with other species increases a species' probability of extinction.
To calculate: The matrix $\bm{A}$ is the asymmetrical adjacency matrix of eat and need interactions with rows corresponding to species and columns corresponding to both species \emph{and} objects.
The number of users (eaters or needers) of each species/object is given by summing across columns resulting in the vector $\bm{x}$, and the number of species or objects eaten or consumed by each species is given by summing across rows resulting in the vector $\bm{y}$.

We calculate the number of species consuming or needing each of a species' possible resources (i.e. the number of species sharing a species' resources), not including itself by $\bm{z} = (\bm{A}\cdot\bm{x}) - \bm{y}$.
Thus the average number of species sharing each species' resources is 
$\bm{ro} = \bm{z}/\bm{y}$ (where division is element-by-element), and the average \emph{proportion} of species sharing each species' resources is given by ${\bm{RO}} = \bm{ro}/(S-1)$ where $S-1$ is the number of potential competitors in the community.

Thus, a species consuming/needing resources that have no other consumers/needers will have an ${\bm{RO}}=0$. In contrast, if all of a species' resources are shared by every other species in the community, ${\bm{RO}} = 1$.
Partial overlap of resource use by other species ranges of course between 0 and 1.

The probability of extinction is assumed to be logistic as a function of ${\bm{RO}}$, where a value close to one will result in a probability of extinction close to one (and zero in the opposing case).
Where the probability of extinction increases, and the slope of the increase are inputs of the model and largely set the steady state of the assembly process.

\section{Example}
If
\(
\bm{A}=
\begin{bmatrix}
    1 & 1 & 1 & 0 & 0 & 0 \\
    1 & 0 & 1 & 0 & 0 & 1 \\
    1 & 1 & 0 & 0 & 0 & 1 \\
    0 & 0 & 0 & 0 & 1 & 0 \\
    1 & 0 & 0 & 0 & 0 & 0
\end{bmatrix}
\)
where rows 1-5 are species, and the last column shows the dependence of each species on a single object, then $\bm{RO} = (0.42,0.42,0.42,0,0.75)$. Resource overlap for species 1-3 is 0.42 because each species consumes 3 resources that are shared by either 2 or 1 additional species. Resource overlap for species 4 is 0 because it eats a single uniquely consumed resource. Resource overlap for species 5 is 0.75 because its single resource is shared by 3 competing species out of the universe of 4 potential competitors.


\end{document}